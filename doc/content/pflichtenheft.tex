\chapter{Pflichtenheft}

\section{Funktionale Anforderungen}

\subsection{Definitionen}
\begin{description}[style=nextline]
	\item[angepasster Stundenplan]
		Stundenplan mit eingearbeiteten Supplierungen 
	\item[tabellarischer Supplierplan]
		Auflistung aller Supplierungen, Ausfälle, etc
	\item[Relevanz bei Ersatzlehrern]
		Ist der Lehrer an diesem Tag nicht in der Schule, so ist er als Supplierlehrer nicht erste Wahl (kursiv oder grau hinterlegt darstellen). Ist er an diesem Tag in der Schule, hat jedoch Unterricht, so ist er nicht als Supplierlehrer einsetzbar, ist er jedoch als Zweitlehrer im selben Unterricht mit dem Absenzlehrer, dann kann er als "Klasse alleine" eingeteilt werden, etc.
\end{description}

\subsection{Supplierungssystem}
Es soll ein System entwickelt werden, dass die Stundenpläne und Supplierungen in digitaler Form speichert. Dazu soll eine Ein­ und Ausgabe der Daten über eine Website und eine App zur Verfügung gestellt werden (siehe Punkt Ausgabe). Weiters wird ein Formular generiert (PDF), das ausgedruckt werden kann.\\
Eingabe über die App nur eingeschränkt und wenn zeitlich möglich.Die Supplierungen und Stundenpläne werden vom Administrator (AV, WL, ...) eingegeben (siehe Punkt Eingabe).

\subsection{News}
News sollen vom Administrator (AV,WL, ...) eingegeben werden (abrufbar über die Website bzw. App).

\subsection{Monitorsystem}
Thin­Clients (z.B.: Raspberry Pi) mit Monitoren sollen mit Daten versorgt werden. Dazu soll nur ein HTML5 kompatibler Browser benötigt werden. Der dementsprechende HTML­Code soll möglichst Auflösungskompatibel sein.
Es soll möglich sein, das auf den Thin­Clients Dargestellte individuell über die Website zu 
konfigurieren.\\

Folgende Möglichkeiten:
\begin{itemize}
	\item 
		angepasster Stundenplan des nächstgelegenen Raumes
	\item
		tabellarischer Supplierplan der Abteilung (mit Informationen bzgl: Magazin und den News)
	\item
		Bild als JPG, PNG oder GIF (Upload über die Website)
	\item
		Video im MP4-Container (Upload über die Website)
	\item
		Uhr
\end{itemize}

\subsection{Authentifizierung}
Authentifizierung erfolgt für die Schüler und Lehrer via LDAP, gilt auch für Monitore (diese müssen sich als Monitore identifizieren. Ohne erfolgreichen Login sind keine Informationen abrufbar.

\subsection{Eingabe}
Administratoren und AVs dürfen Eingaben tätigen. Damit einfache Eingaben auch delegiert werden können muss ein Berechtigungssystem hinterlegt werden.

\begin{description}[style=nextline]
	\item[Lehrer]
		Name, Initialen, Abteilung\\
		Buttons zum Hinzufügen, Editieren und Löschen (LDAP)
	\item[Klassen]
		Name, KV (als Dropdown­Menü), Abteilung (als Dropdown­Menü), Raumbelegung
	\item[Räume]
		Bezeichnung, Abteilung
	\item[Fächer]
		Bezeichnung (Kürzel und Langname)
	\item[Stunden(-pläne)]
		Fach, Lehrer (Dropdown-­Menü; weitere Felder erscheinen bei der Auswahl), Dauer, Raum (Dropdown-­Menü)\\
		\\
		Auswahl der Klasse über ein Menü. Stundenplan aus "Klassen­Sicht". Liste der Wochentage und Buttons zum Hinzufügen, Platzieren, Editieren und Löschen von Stunden im Stundenplan.
	\item[Supplierungen]
		Drei Eingaben:
		\begin{description}[style=nextline]
			\item[fehlende Lehrer]
				Lehrer (Dropdown­-Menü), von­-bis, Grund
			\item[fehlende Klassen]
				Klasse (Dropdown­-Menü), von­-bis, Grund
			\item[Supplierungen]
				Stunde (Dropdown­-Menü), Klasse (Dropdown-­Menü), ausblenden (Check­Box; wenn gesetzt, wird diese Stunde in den angepassten Stundenplänen nicht angezeigt), Supplierlehrer (Dropdown-­Menü; zeigt die Lehrer sortiert und markiert nach Relevanz), Kommentar (Hier wird eingetragen zb: "Mitbetreuung", "Stillbeschäftigung", "entfällt", etc), bestätigen (Check­Box; Eintrag ist erst wirksam, wenn gesetzt)\\
			\\
			Ein Supplierlehrer muss bei Mitbetreuung nicht angegeben werden, da alle anderen Lehrkräfte dieser Stunde, so­wie­so mit dieser verknüpft sind.\\
			\\
			Verschobene Stunden werden als 2 Einträge (einmal "ausgefallen" mit dem "ausblenden"-­Button) und einmal "neu eingefügt" (gekennzeichnet über Kommentar) eingegeben.\\
			\\
			(ev. falls noch Zeit: Wenn ein fehlender Lehrer eingetragen wurde, so werden automatisch alle "Kollisionen" angezeigt.)
		\end{description}
	\item[News]
		Name, Beschreibung, von­bis, Abteilung (Dropdown-­Menü; auch mit Auswahl für die ganze Schule), die News werden nach Ablauf (Bis­Datum) nicht mehr angezeigt, aber nicht gelöscht.
	\item[Monitore]
		Modus (Auswahlliste, siehe Punkt Monitorsystem), falls benötigt: Datei (Upload für Bild, Video)\\
		\\
		Die Monitore melden sich selbst in der DB an, so ist kein Hinzufügen von Monitoren nötig. \\
		Allerdings: Möglichkeit zum Sperren von Einträgen, sollte sich ein Monitor verändern.\\
		\\
		Über Check­Boxen wählt man alle oder einzelne Monitore aus, bei denen man die Konfiguration ändern will. Buttons für alle, keinen und einzelne auswählen.
	 \item[Ausgabe]
	 	Hier gibt es 2 verschiedene Möglichkeiten:
	 	\begin{description}[style=nextline]
	 		\item[Benutzer-Website/App]
	 			Nach Login:\\
	 			Für Schüler und Lehrer wird ein Klassen­/Lehrer­spezifischer angepasster Stundenplan generiert. Über einen Button auf der Startseite kann die Anzeige­Art verändert werden.
	 		\item[Monitore]
	 			siehe Punkt Monitorsystem
	 	\end{description}
	\item[App]
		Es soll eine App für Android, Windows Phone und iOS erstellt werden, die die gleichen Funktionen bietet, wie die Standard­Benutzer­Website (keine Administrativen Funktionen).\\
		\\
		Zusätzlich soll die Benutzer­Website (aufgrund der kompatibilität zu anderen Mobil­Betriebssystemen) auch als mobile Website implementiert werden.
	 \item[Formular]
	 	Das Formular für die Übertragung der Supplierungen in das Abrechnungssystem, wird nach derzeitiger Vorlage generiert. Ein weiteres Formular wäre sinnvoll: Die Auflistung nach fehlendem Lehrer, damit man einen Überblick erhält:\\
	 	\\
	 	\textit{Bsp:}\\
	 	YH fehlend:
\begin{tabbing}
1.6. \= 1. Std. \= TKHF \= 1aHEL \hspace{2em} \= Nz\\
 \> 2. Std. \> TKHF \> 2aHEL \> MT\\
2.6. \> 3. Std. \> LA1 \> 4aHEL \> XY
\end{tabbing}
		...
	\item[Layout]
		Die Eingabeseite/Eingabenmasken, sollen übersichtlich und einfach zu bedienen sein. Das Layout wird der neuen HTL Homepage angepasst (Corporate Design) - als Grundlage dient das FTKL Projekt (Machac, Handle, Wucherer).\\
		\begin{description}[style=nextline]
			\item[Stundenplandesign]
				Als Vorgabe dienen die derzeitigen Raumbeschriftungen der Werkstätten – das Layout wird wieder an das neue Corporate Design angepasst.
			\item[App­Design]
				siehe Corporate Design
			\item[Stundenplaneingabe]
				Am Schuljahresanfang wird der Stundeplan der Abteilung händisch ins SIS übertragen. Die Grundlage für die Eingabe ist der Klassenstundenplan. Es gibt Lehrer, die in anderen Abteilungen eingeteilt sind, es muss für den jeweiligen Administrator möglich sein, auch diese Stunden einzugeben. Die Eingabemaske soll dem Wochenstundenplan angepasst sein (Stunde (1­16) Fach, Klasse, Raum).
			\item[Dokumentation]
				Die Dokumentation wird lt. Vorlage (Mail von Prof. Stecher) ausgeführt. Es sind Bedienungs­ und Serviceanleitungen zu erstellen. Mit diesen Unterlagen muss eine Weiterentwicklung (für andere Diplomanten) und eine Servicesierung durch das Lehrpersonal gewährleistet sein. Der Sourcecode ist sauber zu dokumentieren. Eine Hilfe im Programm im HTML Format ist zu erstellen.\\
				\\
				Ein Projekttagebuch ist zu führen (Beginn des Tasks/Sprints; Zeit und Task; Unterbrechungen; Status)\\
				\\
				Code im Code dokumentieren: doxygen/javadoc
		\end{description}
	\item[Uhranzeige]
		Auf jedem Monitor ist eine Zeitanzeige zu sehen und diese wird dem Design der Anzeigeseite angepasst (Corporate Design).
\end{description}

\section{Schnittstellen}
Es wurde zwar eine Software-Schnittstelle zur verwendeten Schul-Management-Software Untis angedacht, diese Idee wurde aber verworfen, da die Sinnhaftigkeit aufgrund des kommenden Umstiegs der Schule auf eine neue Version in Frage gestellt wird.

\section{Abnahmekriterien}
// TODO

\section{Dokumentationsanforderungen}
// TODO, hint: Javadoc

\section{Qualitätsstandards}
// TODO

\section{Prozessmodell}
// TODO