\subsubsection{Datenbank-Design (Buchberger)}

Bei relationellen Datenbanken, wie das für das Projekt verwendete MySQL, gelten grundsätzliche Richtlinien, um die Konsistenz der Daten zu gewährleisten:

\begin{description}[style=nextline]
	\item[Entitätsintegrität]
		Die Kennung eines Datensatzes (Primärschlüssel) muss auf jeden Fall eindeutig sein. Dies kann durch die MySQL-Eigenschaft "auto increment" erreicht werden.
	\item[Referenzielle Integrität]
		Verweise auf andere Tabellen (Fremdschlüssel) müssen auf Datensätze zeigen, die existieren. Alternativ kann NULL zugewießen werden.\\
		\\
		\textit{Beispiel:} Eine Schulstunde hat einen Lehrer auf den mit einem Fremdschlüssel verwiesen wird. Der Lehrer verlässt die Schule und sein Datensatz wird gelöscht.\\
		\\
		Die Stunde hat weiterhin den gleichen Lehrer, welcher aber nicht mehr existiert.
	\item[Bereichsintegrität]
		Die Werte eines Datenfeldes müssen in einem definierten Bereich liegen.
\end{description}

Weitere wichtige Grundsätze sind:

\begin{description}[style=nextline]
	\item[Redundanzen sind zu vermeiden]
		Anders formuliert: Jede Information soll möglichst nur einmal in der Datenbank vorhanden sein.
		
		Redundanzen sorgen zwangsweise für Probleme, wenn Datensätze aktualisiert werden sollen.
		
		\textit{Beispiel:} Bei einem großen Geschäft wohnen mehrere Kunden aus dem selben Ort. Ändert sich nun die Postleitzahl des Ortes, so müssen alle Datensätze aktualisiert werden.
		
		\textit{Lösung:} Auslagern des Ortes und der Postleitzahl in eine weitere Tabelle und via Fremdschlüssel verknüpfen.\\
		So muss bei Änderung der Postleitzahl nur an einer Stelle geändert werden.
		
		Es stellt sich allerdings die Frage, wie weit die Datenstruktur aufgefächert werden soll.
	\item[Gerechnet wird von der Datenbank]
		Die Sprache, mit der die Abfrage an die Datenbank gestellt wird, soll die Daten nicht mehr nachbearbeiten müssen.\\
		Gründe dafür sind, dass die Datenbank für die Verarbeitung von Daten optimiert ist, und dass es beim sequenziellen Abfragen von Tabellen zu Fällen kommen kann, in denen eine Tabelle breits aktualisiert ist, die verknüpfte Tabelle aber nicht.
\end{description}
