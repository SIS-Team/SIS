\subsubsection{PHP (Handle)}
Die Abkürzung PHP steht für "PHP Hypertext Preprocessor". Es handelt sich hierbei um ein serverseitige Programmiersprache, die vorallem in der Webentwicklung zum Einsatz kommt. Die Syntax ist an Perl und C angelehnt.\\
Als PHP-Module wurden nur php5-mysql sowie php5-ldap verwendet.\\
PHP ist seit Version 5 vollständig objekt-orientiert, wurde aber imperativ/funktional verwendet.\\
\\
Ein PHP Programm kann im Gegensatz zu anderen serverseitigen Programmiersprachen direkt in den HTML-Quelltext der Website eingebunden werden. Gekennzeichnet werden diese eingebetteten Programme mit den PHP-Tags (siehe Programm-Code 3.1).\\
\begin{lstlisting}[style=customPHP, caption={PHP-Tags}]
<?php 
	/* Programm-Code */
?>
\end{lstlisting}
Befindet sich der PHP Code eingebetet in HTML-Queltext, so ignoriert der Interpreter alles, das außerhalb der PHP-Tags steht.\\\\
Eines der großen Vorteile an PHP ist, dass es vollständig serverseitig verarbeitet wird, das heißt am Client wird keine Rechenleistung für das Ausführen des PHP-Codes benötigt.
\paragraph{Funktionsweise}
Der Client fragt am Webserver eine Datei mit der Endung .php an. Anschließend lädt der Webserver die Datei und übergibt diese dem PHP Interpreter. Dieser generiert in den meisten Fällen eine HTML Datei, welche anschließend wieder dem Webserver übergeben wird. Dieser sendet die "fertige" Webseite an den Client. Der PHP Interpreter ist nicht nur auf HTML Dateien begrenzt, es können auch andere Dateitypen, wie Bilder oder PDF Dateien generiert werden. Diese Funktionsweise hat das Problem, dass die Seite bei jedem neuen Aufruf erneut generiert werden muss, dies führt zu einer höheren Auslastung am Webserver. Um dieses Problem zu vermiendern g
\paragraph{Probleme} 
Die Typisierung in PHP ist sehr flexible (dynamisch), so kann einer Variable, die zum Beispiel eine Zahl enthält, eine Zeichenkette, oder ein Array neu zugewiesen werden.\\
Manche Standard-Funktionen in PHP haben numerische Rückgabewerte und geben den bool'schen Wert false zurück, wenn ein Fehler auftritt. Da alle Werte, die nicht 0 sind, laut Definition gleich dem bool'schen true sind, kann es zu Fehlinterpretation des Rückgabewertes kommen. Um solche Situationen so vermeiden, sollte statt auf Wertegleichheit (==) auf Equivalenz (===), das bedeutet in diesem Zusammenhang Werte- und Typgleichheit (Bool != Integer, trotz dynamischer Typisierung), geprüft werden (Beispiel: siehe Programm-Code 3.2).
\begin{lstlisting}[style=customPHP, caption={false}]
<?php 
	$string = "Hallo Welt";
	$position = strpos("H", $string); 
	// H liegt an Position 0
	
	// falsch:
	if ($position == false) {
		echo "Abfrage 1\n";
	}
	// richtig:
		if ($position === false) {
		echo "Abfrage 2\n";
	}
?>
\end{lstlisting}