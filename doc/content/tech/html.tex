\subsubsection{HTML (Weiland)}

HTML steht für "'Hypertext Markup Language"' und bezeichnet eine Auszeichnungssprache. HTML-Dateien werden hauptsächlich für Websites verwendet.


\paragraph{Aufbau}

Eine HTML-Datei ist grundsätzlich immer gleich aufgebaut. Sie besteht aus einem Header in dem unter Anderem der Titel und die Meta-Daten bestimmt werden.
Im Body steht der Inhalt der Datei, welcher angezeigt werden soll. 
\begin{lstlisting}[style=customHTML, caption={HTML-Tags}]
<!DOCTYPE html>
<html> 
	<head>
		/* Datei-Kopf */
	</head>
	<body>
		/* Inhalt der Datei */
	</body>
</html>
\end{lstlisting}

\paragraph{HTML5}
HTML5 ist die aktuellste Version von HTML. Die Entwicklung begann am 29. April 2009 und soll im Jahr 2014 fertiggestellt werden.\\
Im Gegensatz zu früheren Versionen von HTML wird die Wiedergabe von Video- und Audiodateien unterstützt. Jedoch sind aktuell noch nicht alle Browser fähig diese Funktionen zu verwenden.\\
Die unterstützten Formate sind für Videodateien 0gg Theora, MP4(H.264) und WebM(VP8) und für Audiodateien 0gg Vorbis, MP3 und Wav. \\
