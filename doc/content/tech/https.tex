\subsection{HTTP (Weiland)}

Das "'Hypertext Transfer Protocol"' ist ein Protokoll des Application Layers des OSI-Layer-Modells.
\\
HTTP ist ein zustandsloses Protokoll. Anfragen werden stets getrennt behandelt, auch wenn sie vom selben Client stammen. Dies kann durch eine Session geändert werden.
\subsubsection{Verbindungsvorgang}
Zu Beginn wird die Zieladresse in eine IP-Adresse umgewandelt.
Anschließend wird eine TCP-Verbindung  mit dem Server aufgebaut. 
Dann wird eine Anfrage an den Port 80 des Server gesendet.
\\
\textcolor{blue}{GET /index.html HTTP/1.1}
\\
In diesem Fall würde die Datei "'index.html"' mittels HTTP1.1 angefordert werden.
\\
Im Header der Anfrage können außerdem Informationen, wie zum Beispiel der verwendete Webbrowser, enthalten sein.
\\
\textcolor{blue}{User-Agent: Mozilla/5.0 (Windows NT 6.1; WOW64; rv:27.0) Gecko/20100101 Firefox/27.0,}
\\
\\
Wenn die Anfrage erfolgreich bearbeitet wurde, wird an den Client eine Bestätigung gesendet.
\\
\textcolor{blue}{HTTP/1.1 200 OK}
\subsubsection{Statuscodes}
Es gibt für HTTP sechs verschiedene Arten von Satuscodes
\paragraph{1xx}
Diese Statuscodes werden während der Bearbeitung der Anfrage verwendet und dienen der Information.
\paragraph{2xx}
Diese Statuscodes werden benützt , wenn die Anfrage erfolgreich bewältigt wurde.  
\paragraph{3xx}
Diese Codes dienen dazu eine Umleitung ersichtlich zu machen. Bei solch einem Code wird eine Aktion des Clients gefordert, was meist automatisch geschieht.
\paragraph{4xx}
Hiermit werden Fehler gekennzeichnet. Zum Beispiel 404: "Not Found".
\paragraph{5xx}
Diese Codes sollen Server-Fehler kennzeichnen.

\subsubsection{Versionen}
Aktuell sind zwei verschiedene Versionen von HTTP im Einsatz, HTTP/1.0 und HTTP/1.1.
\\
Diese unterscheiden sich insofern, dass bei HTTP/1.0 für jede Anfrage eine neue Verbindung zum Server aufgebaut wird. Dies wirkt sich nachteilig auf die Geschwindigkeit aus, da z.B. auf einer Website mit vielen Bildern, für jedes Bild eine neue Verbindung hergestellt werden muss und diese Verbindungen durch die Eigenschaften von TCP-Verbindungen (z.B. Slow-start ) entsprechend langsam sind.

\subsection{HTTPS (Weiland)}
