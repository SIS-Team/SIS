\subsubsection{LDAP (Buchberger)}

LDAP (Lightweight Directory Access Protocol) ist ein Protokoll für den Zugriff auf Verzeichnis-Datenbanken, wie Microsoft Active Directory, Apple Open Directory, openLDAP oder Novell eDirectory.\\
\\
Verzeichnis-Dienste werden dazu verwendet, um Informationen über Personen oder Rechner-Konfigurationen zu speichern und abzufragen.\\

\paragraph{LDAP-Datenstruktur}

Eine LDAP-Datenstruktur ist Baum-förmig (DIT - Directory Information Tree) aufgebaut.\\
Dies hat den Grund, dass man so geografische und/oder organisatorische Gegebenheiten besser in der Datenbank darstellen kann.
\\
Man unterscheidet zwischen Container-Objekten (enthalten ihrerseits weitere Objekte) und Blatt-Objekten (Sie stellen Enden des Baumes dar).\\
\\
LDAP ist objektorientiert aufgebaut, das heißt, es gibt Objekt-Klassen (wird durch das Schema difiniert), welche Eigenschaften der Objekte vorgeben. Ein LDAP-Objekt besitzt mindestens eine Klasse.\\
\\
Alle Objekte sind durch ihren DN (Distinguished Name) vollständig identifiziert. 
Der DN besteht aus dem Attribut, das das Objekt innerhalb des darüberliegenden Container-Objektes eindeutig macht, zusätzlich zum DN des Container-Objektes.\\
\\
\textit{Beispiel: }\\
uid=mueller,ou=beratung,ou=filliale1,c=at,o=firma\\
\\
Die uid=mueller identifiziert so ein Objekt (vermutlich ein Benutzer) in der ou=beratung. Diese ist wiederum einzigartig in der ou=filliale1, u.s.w.\\
\\
Das lässt sich nun so fortsetzen. Das obestere Container-Objekt wird als Root-Objekt bezeichnet. In den meisten Fällen wird dies mit dem Attribut o (für Organisation) gekennzeichnet.\\
\\
Die Eigenschaften der Objekt sind oft abgekürtzt.\\
\\
\textit{Häufige Abkürtzungen:}\\
\begin{description}[style=nextline]
	\item[o]
		organisationName (Organisation)
	\item[st]
		stateOrProvinceName (Staat/Provinz)
	\item[c]
		country (Land)
	\item[ou]
		organizational unit (Organisations Einheit)
	\item[cn]
		common name (Allgemeiner Name)
	\item[mail]
		e-mail-address (E-Mail-Adresse)
\end{description}

Die Objekt-Klassen definieren die Eigenschaften der Objekt.\\
\\
\textit{Beispiel: } Attribute der Objekt-Klasse inetOrgPerson (Definition des Standard-Schemas initOrgPerson von openLDAP in der Version 1.4.2.6)\\

\begin{description}[style=nextline]
	\item[erbt von]
		organizationalPerson
		\begin{description}[style=nextline]
			\item[erbt von]
				person
				\begin{description}[style=nextline]
					\item[erbt von]
						top	(abstrakte Klasse; keine Attribute)
					\item[!sn]
						sur name (Nachname)
					\item[!cn]
						common name (Allgemeiner Name)
					\item[userPassword]
						Benutzerpasswort
					\item[telephoneNumber]
						Telefon Nummer
					\item[seeAlso]
						Verweiß (als dn)
					\item[description]
						Beschreibung
				\end{description}
			\item[title]
				Titel
			\item[x121Address]
				wurde für das x.121-Protokoll verwendet
			\item[registeredAddress]
				Registrierungs Adresse
			\item[destinationIndicator]
				wurde für Telegram-Services verwendet
			\item[preferredDeliveryMethod]
				Bevorzugte Liefer Methode
			\item[telexNumber]
				wurde für Telex verwendet
			\item[teletexTerminalIdentifier]
				wurde für Teletex verwendet
			\item[telephoneNumber]
				Telefon Nummer
			\item[internationaliSDNNumber]
				ISDN Nummer
			\item[facsimileTelephoneNumber]
				Fax Nummer
			\item[street]
				Staßen Adresse
			\item[postOfficeBox]
				Postfach
			\item[postalCode]
				Postleitzahl
			\item[postalAddress]
				Post Adresse
			\item[physicalDeliveryOfficeName]
				Postbüro Name
			\item[ou]
				organisation unit (Organisations Einheit)
			\item[st]
				stateOrProvinceName (Staat/Provinz)
			\item[l]
				localityName (Ort/Ortschaft)
		\end{description}
	\item[audio]
		Audio
	\item[businessCategory]
		Geschäfts-Kategorie
	\item[carLicense]
		Fahrzeug-Lizenz
	\item[departmentNumber]
		Abteilung innerhalb eines Unternehmens
	\item[displayName]
		bevorzugter Anzeigename
	\item[employeeNumber]
		Angestellten Nummer
	\item[employeeType]
		Anstellungsart
	\item[givenName]
		Vorname(n)
	\item[homePhone]
		Heim-Telefon-Nummer
	\item[homePostalAddress]
		Heim-Andresse
	\item[initials]
		Initialen
	\item[jpegPhoto]
		JPEG Foto
	\item[labeledURI]
		Uniforme Resourcen ID mit optimaler Bezeichnung
	\item[mail]
		e-mail-address (E-Mail-Adresse)
	\item[manager]
		dn des Managers
	\item[mobile]
		mobile Telefon-Nummer
	\item[o]
		organisationName (Organisation)
	\item[pager]
		Telefon-Nummer des Pagers
	\item[photo]
		Foto
	\item[roomNumber]
		Raum Nummer
	\item[secretary]
		dn des Sekretärs
	\item[uid]
		Benutzer Kennung
	\item[userCertificate]
		Beuntzer Zertifikat		
	\item[x500uniqueIdentifier]
		ID für das X.500 Protokoll
	\item[preferredLanguage]
		bevorzugte Sprache
	\item[userSMIMECertificate]
		SignedData für S/MIME
	\item[userPKCS12]
		PKCS \#12 PFX PDU für den Austausch von persönlichen Identitäts-Informationen
\end{description}

\paragraph{LDAP-Suche}

Für die Suche in einer LDAP-Datenstruktur wird als erstes ein baseDN festgelegt, dieser entspricht dem Objekt, ab dem im Baum gesucht werden soll. Weiters wird der Scope festgelegt, hierbei gibt es 3 verschiedene Möglichkeiten:

\begin{description}[style=nextline]
	\item[base]
		Hierbei wird nur das Objekt mit dem baseDN durchsucht.
	\item[one]
		Eine Ebene unterhalb des baseDN wird gesucht.
	\item[sub]
		Alle darunter liegende Objekte sowie das, durch den baseDN referenzierte, werden durchsucht.
\end{description}
Die Suchkriterien werden in der Polnischen Notation (Präfixnotation) geschrieben und dürfen Wildcards enthalten.\\
\\
\textit{Beispiel:}\\
baseDN: ou=STUDENTS,o=HTLinn\\
scope: sub\\
(\&(l=5*H*)(mail=Andreas*)(cn=2009*))\\
\\
Diese Suchanfrage würde alle Objecte liefern, die
\begin{itemize}
	\item
		unterhalb von ou=STUDENTS,o=HTLinn liegen 
	\item	
		und die die Eigenschaften mail, l und cn enthalten
	\item	
		und bei denen der cn mit 2009 beginnt 
	\item	
		und deren E-Mail-Adresse mit Andreas beginnt 
	\item	
		und deren Ort-Attribut mit 5 beginnt und ein H enthält.
\end{itemize}